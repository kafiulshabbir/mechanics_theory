\section{Surface direction and pressure}
	In the the prism.
	
	The forces are given by:
	
	\eqb
		F_x = p_x \Delta y d
	\eqe
	
	\eqb
		F_y = p_y \Delta x d
	\eqe
	
	\eqb
		F_n = p_n \Delta s d
	\eqe
	
	Relation between the surface areas:
	
	\eqb
		\Delta x = \Delta s \cos(\theta)
	\eqe
	
	\eqb
		\Delta y = \Delta s \sin(\theta)
	\eqe
	
	Forces in terms of $\Delta s$:
	
	\eqb
		F_x = p_x \Delta s \sin(\theta) d
	\eqe
	
	\eqb
		F_y = p_y \Delta s \cos(\theta) d
	\eqe
	
	\eqb
		F_n = p_n \Delta s d
	\eqe
	
	Balance of forces:
	\eqb
		F_x = F_n \sin(\theta)
	\eqe
	
	\eqb
		F_y = F_n \cos(\theta)
			+ \rho \frac{1}{2} \Delta x \Delta y d  g
	\eqe
	
	Balance of pressures:
	\eqb
		p_x \Delta s \sin(\theta) d
			= (p_n \Delta s d) \sin(\theta)
	\eqe
	
	\eqb
		p_y \Delta s \cos(\theta) d
			= (p_n \Delta s d) \cos(\theta)
			+ \rho \frac{1}{2} \Delta x \Delta y d  g
	\eqe
	
	Simplifying:
	\eqb
		p_x = p_n 
	\eqe
	
	\eqb
		p_y = pP_n
			+ \rho \frac{1}{2}
			\frac{\Delta x \Delta y}{\Delta s \cos(\theta)}  g
	\eqe
	
	\eqb
		p_y = p_n + \rho \frac{1}{2} \Delta y  g
	\eqe
	
	For small prism:
	\eqb
		\Delta y \to 0
	\eqe

	\eqb
		p_y = p_n
	\eqe
	
	Pressure does not depend on the orientation of the surface.
	
\section{Difference of pressures between two points}
	Consider a cube.
	
	Along the $x-axis$, the forces:
	\eqb
		F_x(x) = p(x) dy dz 
	\eqe
	
	\eqb
		F_x(x + dx) = -p(x + dx) dy dz
	\eqe
	
	The net force:
	\eqb
		dF_x = F_x(x) + F_x(x + dx)
	\eqe
	
	\eqb
		dF_x = p(x) dy dz  -p(x + dx) dy dz
	\eqe
	
	\eqb
		dF_x = -(p(x + dx) - p(x)) dy dz
	\eqe
	
	\eqb
		dF_x = -\frac{\partial p}{\partial x} dx dy dz
	\eqe
	
	The infinitesimal volume is given by:
	\eqb
		dV = dx dy dz
	\eqe
	
	Let $f$ be force per unit volume:
	\eqb
		f = \frac{dF}{dV}
	\eqe
	
	Then:
	\eqb
		f_x = -\frac{\partial p}{\partial x}
	\eqe
	
	\eqb
		\vec{f} = -\left( 
		 \frac{\partial p}{\partial x} \vec{i}
			+ \frac{\partial p}{\partial y} \vec{j}
			+ \frac{\partial p}{\partial z} \vec{k}
			\right)
	\eqe
	
	\eqb \label{eq:f-is-grad-of-p}
		\vec{f_{p}} = -\nabla p
	\eqe
	
	\eqb
		p = p(x, y, z)
	\eqe
	
	\eqb
		dp = \frac{\partial p}{\partial x} dx
			+ \frac{\partial p}{\partial y} dy
			+ \frac{\partial p}{\partial z} dz
	\eqe
	
	\eqb
		d\vec{l} = dx \vec{i} + dy \vec{j} + dz \vec{k}
	\eqe
	
	\eqb
		dp = (\nabla p) \cdot d\vec{l}
	\eqe
	
	Taking dot product of $d\vec{l}$ on both sides of equation \ref{eq:f-is-grad-of-p}
	\eqb
		\vec{f_{p}} \cdot d\vec{l} = -(\nabla p) \cdot d\vec{l} 
	\eqe
	
	\eqb
		\vec{f_{p}} \cdot d\vec{l} = -dp 
	\eqe
	
	Force due to gravity per unit volume:
	\eqb
		\vec{dF_{g}} = \rho dV \vec{g}
	\eqe
	
	\eqb
		\vec{f_g} = \rho  \vec{g}
	\eqe
	
	The force due to the pressure difference is balanced by the weight:
	
	\eqb
		\vec{f_p} = -\vec{f_{g}}
	\eqe
	

	Taking integral on both sides:
	\eqb
		\int_{\vec{r_1}}^{\vec{r_2}} \vec{f_{g}} \cdot d\vec{l}
			= \int_{\vec{r_1}}^{\vec{r_2}} dp 
	\eqe
	
	\eqb
		p(\vec{r_2}) - p(\vec{r_1}) = \int_{\vec{r_1}}^{\vec{r_2}} \vec{f_{g}} \cdot d\vec{l}
	\eqe
	
	Because of this, the fluid at the bottom of arbitary shaped tank is the same.
	
	\eqb
		p(\vec{r_2}) - p(\vec{r_1}) = \int_{\vec{r_1}}^{\vec{r_2}} \rho \vec{g} \cdot d\vec{l}
	\eqe
	
	For constant density and gravity $\vec{g} = -g\vec{j}$:
	\eqb
		p(\vec{r_2}) - p(\vec{r_1}) = \rho g \int_{\vec{r_1}}^{\vec{r_2}} -\vec{j} \cdot d\vec{l}
	\eqe
	
	\eqb
		p(\vec{r_2}) - p(\vec{r_1}) = -\rho g \int_{\vec{r_1}}^{\vec{r_2}} y
	\eqe
	
	\eqb
		p_2 - p_1 = -\rho g (y_2 - y_1)
	\eqe
	
	\eqb
		p + \rho g y = const
	\eqe
	
	
